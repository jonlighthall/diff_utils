\section{Mathematical Foundation}
\label{sec:math}

\subsection{Machine Precision Constants}

\subsubsection{Single-Precision Epsilon}

Single-precision floating-point numbers (IEEE 754 binary32) have 23 mantissa bits:

\begin{equation}
    \epsilon_{\text{single}} = 2^{-23} \approx 1.19 \times 10^{-7}
\end{equation}

This represents the smallest resolvable \emph{relative} difference for normalized single-precision values.

\subsubsection{Double-Precision Epsilon}

Double-precision floating-point numbers (IEEE 754 binary64) have 52 mantissa bits:

\begin{equation}
    \epsilon_{\text{double}} = 2^{-52} \approx 2.22 \times 10^{-16}
\end{equation}

\subsection{Domain-Specific Threshold Derivations}

\subsubsection{Ignore Threshold}

For transmission loss in underwater acoustics, the relationship between pressure and transmission loss is:

\begin{equation}
    TL = -20 \log_{10}\left(\frac{p}{p_{\text{ref}}}\right)
\end{equation}

The minimum representable pressure in single-precision arithmetic (relative to reference pressure) is $\epsilon_{\text{single}}$. The corresponding transmission loss is:

\begin{align}
    TL_{\text{ignore}} &= -20 \log_{10}(\epsilon_{\text{single}}) \\
    &= -20 \log_{10}(2^{-23}) \\
    &= -20 \times (-23) \log_{10}(2) \\
    &= 460 \log_{10}(2) \\
    &\approx 138.47 \text{ dB}
\end{align}

\textbf{Physical interpretation:} Transmission loss values above 138.47 dB represent acoustic pressures that cannot be reliably represented in single-precision arithmetic.

\subsubsection{Marginal Threshold}

Based on acoustic propagation research \cite{doi:10.23919/OCEANS.2009.5422312}, transmission loss values exceeding 110 dB are weighted to zero in practical sonar calculations. This threshold defines the boundary of operational significance.

\begin{equation}
    TL_{\text{marginal}} = 110 \text{ dB}
\end{equation}

\subsection{LSB and Sub-LSB Criteria}

\subsubsection{Least Significant Bit}

For a value printed with $p$ decimal places, the LSB is:

\begin{equation}
    \LSB = 10^{-p}
\end{equation}

This represents the minimum step size in the printed representation.

\subsubsection{Half-LSB (big\_zero)}

The rounding uncertainty for a value at precision $p$ is:

\begin{equation}
    \text{big\_zero} = \frac{\LSB}{2} = \frac{10^{-p}}{2}
\end{equation}

\textbf{Information-theoretic justification:} A value printed as $v$ with precision $p$ represents the interval:

\begin{equation}
    \left[v - \frac{\LSB}{2}, v + \frac{\LSB}{2}\right)
\end{equation}

\subsubsection{Sub-LSB Detection with Floating-Point Tolerance}

To handle floating-point representation errors, the sub-LSB criterion includes a relative tolerance:

\begin{equation}
    \text{sub-LSB} \iff \left(|\Delta v| < \text{big\_zero}\right) \vee \left(||\Delta v| - \text{big\_zero}| < \epsilon_{FP} \cdot \max(|\Delta v|, \text{big\_zero})\right)
\end{equation}

where $\epsilon_{FP} = 10^{-12}$ is the floating-point comparison tolerance.

\subsection{Percent Error Calculation}

For non-trivial differences, percent error is computed using the second file as reference:

\begin{equation}
    E_{\%} = 100 \times \frac{|\Delta v|}{|v_2|}
\end{equation}

\textbf{Special case:} When $|v_2| \leq \epsilon_{\text{single}}$, the percent error is undefined and reported as $\infty$ (sentinel value).
