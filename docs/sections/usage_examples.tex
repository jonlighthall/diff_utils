\section{Usage Examples}
\label{sec:examples}

\subsection{Basic Usage}

\subsubsection{Default Comparison (threshold = 0)}

\begin{lstlisting}[language=bash]
$ uband_diff file1.txt file2.txt
\end{lstlisting}

Reports all non-trivial differences (maximum sensitivity mode).

\subsubsection{Absolute Threshold}

\begin{lstlisting}[language=bash]
$ uband_diff -t 0.01 file1.txt file2.txt
\end{lstlisting}

Ignores differences below 0.01 (after rounding to common precision).

\subsubsection{Percent Threshold}

\begin{lstlisting}[language=bash]
$ uband_diff -t -2 file1.txt file2.txt
\end{lstlisting}

Ignores differences below 2\% (negative value activates percent mode).

\subsection{Output Interpretation}

\subsubsection{Successful Comparison}

\begin{verbatim}
SUMMARY (rounded to minimum decimal places):
Files compared: 960 values total

Raw differences: 0 non-zero, 960 zero

After rounding:
  Trivial differences (formatting only): 0
  Non-trivial differences: 0

COMPARISON: Files are EQUIVALENT (0 non-trivial differences)
\end{verbatim}

\textbf{Interpretation:} Files are bitwise identical at printed precision.

\subsubsection{Sub-LSB Differences}

\begin{verbatim}
SUMMARY (rounded to minimum decimal places):
Files compared: 960 values total

Raw differences: 120 non-zero, 840 zero
  Maximum non-zero difference: 0.05 (at 1 decimal places)

After rounding:
  Trivial differences (sub-LSB or rounded=0): 120
  Non-trivial differences: 0

COMPARISON: Files are EQUIVALENT (0 non-trivial differences)
\end{verbatim}

\textbf{Interpretation:} All differences are sub-LSB (formatting-induced).

\subsubsection{Normal Differences}

\begin{verbatim}
SUMMARY (rounded to minimum decimal places):
Files compared: 960 values total

Raw differences: 150 non-zero, 810 zero
  Maximum non-zero difference: 1.25 (at 2 decimal places)

After rounding:
  Trivial differences: 50
  Non-trivial differences: 100
    Maximum non-trivial difference: 1.20 (at 2 decimal places)

Machine precision classification (ε = 1.19e-7):
  Subnormal differences: 30
  Normal differences: 70
    Maximum normal difference: 1.20 (2dp)
    Maximum percent error: 5.6%

  Zero-weighted TL differences [110, 138 dB]: 10
  Non-zero-weighted differences: 60
    Critical differences (> 10.0): 0
    Non-critical differences: 60

COMPARISON: Files DIFFER (60 non-zero-weighted differences, 6.25% of total)
\end{verbatim}

\textbf{Interpretation:} 60 non-zero-weighted differences detected (6.25\% of 960 values). No critical failures. Files differ but within expected numerical variability.

\subsection{Exit Codes}

\begin{description}
    \item[0] Files are equivalent (no non-zero-weighted differences)
    \item[1] Files differ (non-zero-weighted differences found)
    \item[2] File access error or structural mismatch
    \item[3] Critical difference detected (hard failure)
\end{description}

\subsection{Integration with Automated Testing}

\subsubsection{Makefile Integration}

\begin{lstlisting}[language=make]
test: test_regression
	@uband_diff -t 0.01 reference.txt output.txt
	@echo "Validation passed"

.PHONY: test
\end{lstlisting}

\subsubsection{CI/CD Pipeline}

\begin{lstlisting}[language=bash]
#!/bin/bash
# run_validation.sh

MODEL_OUTPUT="output/transmission_loss.txt"
REFERENCE="data/reference/tl_baseline.txt"
THRESHOLD=0.05

if uband_diff -t ${THRESHOLD} ${REFERENCE} ${MODEL_OUTPUT}; then
    echo "[PASS] Model validation PASSED"
    exit 0
else
    echo "[FAIL] Model validation FAILED"
    exit 1
fi
\end{lstlisting}
