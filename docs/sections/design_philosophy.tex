\section{Design Philosophy}
\label{sec:design}

\subsection{Core Principles}

The \ubdiff{} comparison algorithm is built on three fundamental principles:

\subsubsection{Progressive Refinement}

The algorithm implements a hierarchical classification pipeline where each level partitions one set into exactly two subsets, with one subset undergoing further subdivision at the next level. This creates a binary decision tree that progressively refines the classification from raw numeric comparison to domain-specific significance assessment.

\textbf{Mathematical structure:}
\begin{align*}
    \text{Level } n: \quad S_n &= A_n \cup B_n \\
    \text{Level } n+1: \quad B_n &= A_{n+1} \cup B_{n+1}
\end{align*}

This ensures that at each level, all elements remain accounted for, maintaining rigorous statistical invariants.

\subsubsection{Precision Awareness}

Differences are evaluated in the context of the \emph{printed precision} of numerical values. A value printed as \texttt{30.8} (1 decimal place) implicitly represents the interval $[30.75, 30.85)$, carrying an inherent uncertainty of $\pm 0.05$.

\textbf{Key insight:} The minimum representable difference at precision $p$ decimal places is:
\begin{equation}
    \LSB = 10^{-p}
\end{equation}

Differences smaller than $\LSB/2$ are \emph{sub-LSB} and cannot be distinguished at the given precision.

\subsubsection{Domain Specificity}

For transmission loss (TL) acoustic data, the algorithm incorporates physics-based thresholds:

\begin{itemize}
    \item \textbf{Ignore threshold} ($\approx 138$ dB): Values exceeding this represent pressures below single-precision arithmetic limits
    \item \textbf{Marginal threshold} (110 dB): Based on operational research, TL values above 110 dB are weighted to zero in propagation analysis \cite{doi:10.23919/OCEANS.2009.5422312}
\end{itemize}

These domain constraints are bypassed for range/distance data (column 0), which is automatically detected via monotonicity and fixed-delta criteria.

\subsection{Architectural Design Pattern}

\ubdiff{} implements a \textbf{pipeline architecture} with separation of concerns:

\begin{description}
    \item[FileReader] Parses files, extracts column values, detects precision
    \item[DifferenceAnalyzer] Applies discrimination logic, tracks statistics
    \item[FileComparator] Orchestrates comparison, generates summaries
    \item[Main] Command-line interface, threshold configuration
\end{description}

This modular design enables independent testing of each component and facilitates future extensions.

\subsection{Fail-Fast vs. Fail-Complete}

The implementation uses a \textbf{hybrid strategy}:

\begin{itemize}
    \item \textbf{Fail-fast}: First critical difference triggers immediate error reporting
    \item \textbf{Fail-complete}: Processing continues to gather complete statistics
\end{itemize}

This provides both rapid feedback (useful in automated testing) and comprehensive diagnostics (useful in debugging).
