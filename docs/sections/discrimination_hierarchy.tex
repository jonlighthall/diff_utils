\section{Six-Level Discrimination Hierarchy}
\label{sec:hierarchy}

\subsection{Overview}

The discrimination algorithm implements a six-level binary partition hierarchy. Each level applies specific criteria to progressively categorize differences from raw comparison through domain-specific significance assessment.

\textbf{Hierarchical invariants at each level:}

\begin{align*}
    \text{Level 0:} \quad & \text{total} = \Sigma(\text{lines} \times \text{columns}) \\
    \text{Level 1:} \quad & \text{total} = \text{zero} + \text{non-zero} \\
    \text{Level 2:} \quad & \text{non-zero} = \text{trivial} + \text{non-trivial} \\
    \text{Level 3:} \quad & \text{non-trivial} = \text{insignificant} + \text{significant} \\
    \text{Level 4:} \quad & \text{significant} = \text{marginal} + \text{non-marginal} \\
    \text{Level 5:} \quad & \text{non-marginal} = \text{critical} + \text{non-critical}
\end{align*}

\subsection{Level 0: Structure Validation}

\textbf{Purpose:} Verify file structure compatibility and count total elements

\textbf{Partition:}
\begin{equation}
    \text{total\_elements} = \text{countable\_elements} + \text{structural\_mismatches}
\end{equation}

\textbf{Implementation:} \texttt{FileComparator::compare\_files()}

\textbf{Decision Logic:}
\begin{lstlisting}[language=C++]
total_elements = sum(lines_per_group * columns_per_group)
\end{lstlisting}

\textbf{Failure Behavior:} If element counts differ between files, comparison continues up to the structural divergence point but is marked as failed.

\subsection{Level 1: Raw Difference Detection}

\textbf{Purpose:} Distinguish identical elements from those with any measurable difference

\textbf{Partition:}
\begin{equation}
    \text{total} = \text{zero\_diff} + \text{non\_zero\_diff}
\end{equation}

\textbf{Decision Rule:}
\begin{lstlisting}[language=C++]
raw_diff = |value1 - value2|
non_zero = (raw_diff > thresh.zero)  // thresh.zero = epsilon_single
\end{lstlisting}

\textbf{Implementation:} \texttt{DifferenceAnalyzer::process\_raw\_values()}

\textbf{Counters:} \texttt{counter.diff\_non\_zero}, \texttt{differ.max\_non\_zero}

\textbf{Significance:} If non-zero count is zero, files are bitwise equivalent (within epsilon). This provides functionality similar to \texttt{diff} but with epsilon tolerance.

\subsection{Level 2: Precision-Based Trivial Detection}

\textbf{Purpose:} Separate format-driven differences from substantive numerical differences

\textbf{Partition:}
\begin{equation}
    \text{non\_zero} = \text{trivial} + \text{non\_trivial}
\end{equation}

\textbf{Decision Rule:}
\begin{lstlisting}[language=C++]
LSB = pow(10.0, -min_dp)  // Least Significant Bit
big_zero = LSB / 2.0       // Half-ULP criterion
rounded_diff = round_to_decimals(raw_diff, min_dp)

constexpr double FP_TOLERANCE = 1e-12;
bool sub_lsb = (raw_diff < big_zero) ||
    (abs(raw_diff - big_zero) < FP_TOLERANCE * max(raw_diff, big_zero));

trivial = (rounded_diff == 0.0) || sub_lsb;
\end{lstlisting}

\textbf{Implementation:} \texttt{DifferenceAnalyzer::process\_rounded\_values()}

\textbf{Counters:} \texttt{counter.diff\_trivial}, \texttt{counter.diff\_non\_trivial}

\textbf{Example:} Comparing \texttt{30.8} (1dp) vs \texttt{30.85} (2dp):
\begin{align*}
    \LSB &= 10^{-1} = 0.1 \\
    \text{big\_zero} &= 0.05 \\
    \text{raw\_diff} &= |30.8 - 30.85| = 0.05 \\
    \text{Test:} \quad 0.05 &\leq 0.05 \quad \checkmark \\
    \text{Classification:} \quad & \textbf{TRIVIAL} \text{ (sub-LSB difference)}
\end{align*}

\textbf{Percent Error Tracking:} For non-trivial differences:
\begin{equation}
    E_{\%} = \begin{cases}
        100 \times \frac{\text{raw\_diff}}{|v_2|} & \text{if } |v_2| > \epsilon_{\text{single}} \\
        \infty & \text{otherwise}
    \end{cases}
\end{equation}

\subsection{Level 3: Significance Assessment}

\textbf{Purpose:} Among non-trivial differences, separate numerically meaningful from those dominated by machine precision

\textbf{Partition:}
\begin{equation}
    \text{non\_trivial} = \text{insignificant} + \text{significant}
\end{equation}

\textbf{Decision Rule:}
\begin{lstlisting}[language=C++]
// Check domain threshold (skip for range data)
both_above_ignore = (value1 > thresh.ignore) &&
                   (value2 > thresh.ignore) &&
                   !is_range_data;

if (both_above_ignore) {
    classification = INSIGNIFICANT;  // Beyond numerical reliability
} else {
    // Apply significance threshold
    if (thresh.significant_is_percent) {
        // Percent mode
        ref = abs(value2);
        if (ref <= thresh.zero) {
            exceeds = (rounded_diff > thresh.zero);
        } else {
            exceeds = (rounded_diff / ref) > thresh.significant_percent;
        }
    } else if (thresh.significant == 0.0) {
        // Sensitive mode: all non-trivial = significant
        exceeds = true;
    } else {
        // Standard mode: absolute threshold
        exceeds = (rounded_diff > thresh.significant);
    }

    classification = exceeds ? SIGNIFICANT : INSIGNIFICANT;
}
\end{lstlisting}

\textbf{Implementation:} \texttt{DifferenceAnalyzer::process\_rounded\_values()}

\textbf{Counters:} \texttt{counter.diff\_insignificant}, \texttt{counter.diff\_significant}, \texttt{counter.diff\_high\_ignore}

\textbf{Special Modes:}
\begin{description}
    \item[Percent Mode] Enabled when \texttt{thresh.significant < 0} (e.g., -10 for 10\%)
    \item[Sensitive Mode] When \texttt{thresh.significant == 0.0}, all non-trivial below ignore are significant
    \item[Range Data] Column 0 bypasses TL-specific thresholds if detected as monotonic range values
\end{description}

\subsection{Level 4: Marginal vs Non-Marginal}

\textbf{Purpose:} Distinguish operationally marginal from truly non-marginal differences

\textbf{Partition:}
\begin{equation}
    \text{significant} = \text{marginal} + \text{non\_marginal}
\end{equation}

\textbf{Decision Rule:}
\begin{lstlisting}[language=C++]
if (!is_range_data &&
    value1 > thresh.marginal && value1 < thresh.ignore &&
    value2 > thresh.marginal && value2 < thresh.ignore) {
    classification = MARGINAL;  // Both in [110, 138.47] dB band
} else {
    classification = NON_MARGINAL;
}
\end{lstlisting}

\textbf{Implementation:} \texttt{DifferenceAnalyzer::process\_rounded\_values()}

\textbf{Counter:} \texttt{counter.diff\_marginal}

\textbf{Domain Context:} For TL data, the marginal band [110, 138.47] dB represents numerically valid but operationally insignificant values.

\subsection{Level 5: Critical vs Non-Critical}

\textbf{Purpose:} Detect catastrophically large differences indicating potential model failure

\textbf{Partition:}
\begin{equation}
    \text{non\_marginal} = \text{critical} + \text{non\_critical}
\end{equation}

\textbf{Decision Rule:}
\begin{lstlisting}[language=C++]
if (!is_range_data &&
    rounded_diff > thresh.critical &&
    value1 <= thresh.ignore && value2 <= thresh.ignore) {
    classification = CRITICAL;
    flags.has_critical_diff = true;
    flags.error_found = true;
} else {
    classification = NON_CRITICAL;
}
\end{lstlisting}

\textbf{Implementation:} Early check in \texttt{DifferenceAnalyzer::process\_difference()}

\textbf{Counter:} \texttt{counter.diff\_critical}

\textbf{Behavior:} Upon first critical difference, \texttt{print\_hard\_threshold\_error()} reports it. Processing continues for complete statistics, but exit code is set to failure.

\subsection{Complete Hierarchy Diagram}

\begin{figure}[h]
\centering
\small
\begin{verbatim}
TOTAL ELEMENTS (Level 0)
|-- ZERO DIFF (Level 1)
`-- NON-ZERO DIFF (Level 1)
    |-- TRIVIAL (Level 2: sub-LSB or rounded=0)
    `-- NON-TRIVIAL (Level 2)
        |-- INSIGNIFICANT (Level 3)
        `-- SIGNIFICANT (Level 3)
            |-- MARGINAL (Level 4)
            `-- NON-MARGINAL (Level 4)
                |-- CRITICAL (Level 5)
                `-- NON-CRITICAL (Level 5)
\end{verbatim}
\caption{Six-level discrimination hierarchy}
\label{fig:hierarchy}
\end{figure}
