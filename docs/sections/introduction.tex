\section{Introduction}
\label{sec:introduction}

\subsection{Motivation}

Numerical file comparison is a fundamental requirement in scientific computing, computational physics, and software validation. Traditional text comparison tools (e.g., \texttt{diff}) perform bitwise comparison, which fails for numerical data where:

\begin{itemize}
    \item Values may be formatted with different precisions (e.g., \texttt{30.8} vs \texttt{30.85})
    \item Floating-point arithmetic introduces platform-dependent rounding
    \item Compiler optimizations produce numerically equivalent but not bitwise identical results
    \item Cross-language implementations (Fortran, C++, Python) generate format variations
\end{itemize}

\subsection{Problem Statement}

Given two numerical files, determine whether they are \emph{numerically equivalent} within:
\begin{enumerate}
    \item The precision at which values are printed
    \item User-specified significance thresholds
    \item Domain-specific operational constraints
    \item Machine precision limitations
\end{enumerate}

\subsection{Application Domain}

The \ubdiff{} utility was developed primarily for underwater acoustic propagation modeling, specifically for comparing transmission loss (TL) output files from acoustic models. However, its precision-aware comparison methodology is applicable to any domain requiring robust numerical validation.

\textbf{Typical use cases include:}
\begin{itemize}
    \item Regression testing after code refactoring
    \item Cross-platform validation (x86 vs ARM, Windows vs Linux)
    \item Compiler optimization verification (GCC vs Clang, -O2 vs -O3)
    \item Multi-language code validation (Fortran reference vs C++ port)
    \item Scientific reproducibility verification
\end{itemize}

\subsection{Design Goals}

\begin{enumerate}
    \item \textbf{Precision Awareness:} Respect the printed precision of numerical values
    \item \textbf{Cross-Platform Robustness:} Tolerate formatting differences that preserve numerical equivalence
    \item \textbf{Domain Specificity:} Support physics-based thresholds for specialized applications
    \item \textbf{Progressive Refinement:} Hierarchical classification from raw differences to domain significance
    \item \textbf{Rigorous Accounting:} Maintain mathematical invariants across all discrimination levels
    \item \textbf{Usability:} Provide clear, actionable output with minimal configuration
\end{enumerate}

\subsection{Document Organization}

This report is organized as follows:

\begin{description}
    \item[Section~\ref{sec:design}] Design philosophy and architectural principles
    \item[Section~\ref{sec:math}] Mathematical foundation and threshold derivations
    \item[Section~\ref{sec:hierarchy}] Six-level discrimination hierarchy
    \item[Section~\ref{sec:sublsb}] Sub-LSB detection theory and implementation
    \item[Section~\ref{sec:implementation}] Code structure and key algorithms
    \item[Section~\ref{sec:validation}] Test suite and validation methodology
    \item[Section~\ref{sec:examples}] Usage examples and output interpretation
    \item[Section~\ref{sec:conclusion}] Summary and future enhancements
\end{description}
