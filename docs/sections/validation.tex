\section{Validation and Testing}
\label{sec:validation}

\subsection{Unit Test Suite}

The implementation includes comprehensive unit tests covering all discrimination levels and edge cases:

\subsubsection{Test Categories}

\begin{enumerate}
    \item \textbf{Semantic Invariants} (\texttt{test\_semantic\_invariants.cpp})
    \begin{itemize}
        \item Counter summation invariants at each level
        \item Partition completeness verification
        \item Mutual exclusivity of categories
    \end{itemize}

    \item \textbf{Sub-LSB Boundary} (\texttt{test\_sub\_lsb\_boundary.cpp})
    \begin{itemize}
        \item Exact half-LSB difference (30.8 vs 30.85)
        \item Sub-LSB at multiple precision levels
        \item Supra-LSB differences remain non-trivial
        \item Mixed sub/supra-LSB cases
        \item Cross-platform formatting equivalence
        \item Sub-LSB with non-zero thresholds
    \end{itemize}

    \item \textbf{Percent Threshold} (\texttt{test\_percent\_threshold.cpp})
    \begin{itemize}
        \item Percent mode activation (negative threshold)
        \item Reference value (value2) usage
        \item Near-zero reference handling (INF sentinel)
        \item Mixed percent/absolute thresholding
    \end{itemize}

    \item \textbf{Sensitive Threshold} (\texttt{test\_sensitive\_threshold.cpp})
    \begin{itemize}
        \item Zero threshold mode (maximum sensitivity)
        \item All non-trivial classified as significant
        \item Ignore threshold still applies
    \end{itemize}

    \item \textbf{TL-Specific} (\texttt{test\_trivial\_exclusion.cpp}, \texttt{test\_unit\_mismatch.cpp})
    \begin{itemize}
        \item Marginal band classification [110, 138] dB
        \item Ignore threshold exclusion (>138 dB)
        \item Range data bypass
        \item Unit mismatch detection
    \end{itemize}
\end{enumerate}

\subsection{Pi Precision Test Suite}

A domain-agnostic validation using mathematical constant $\pi$:

\subsubsection{Test Methodology}

\begin{enumerate}
    \item Fortran program calculates $\pi$ using Machin's formula:
    \begin{equation}
        \frac{\pi}{4} = 4 \arctan\left(\frac{1}{5}\right) - \arctan\left(\frac{1}{239}\right)
    \end{equation}

    \item Output $\pi$ with increasing precision (0dp to 14dp):
    \begin{align*}
        0\text{dp:} & \quad 3 \\
        1\text{dp:} & \quad 3.1 \\
        2\text{dp:} & \quad 3.14 \\
        & \vdots \\
        14\text{dp:} & \quad 3.14159265358979
    \end{align*}

    \item Compare all precision pairs with \texttt{threshold=0}

    \item Verify all pairs classified as equivalent (sub-LSB)
\end{enumerate}

\subsubsection{Test Results}

\begin{itemize}
    \item \checkmark{} Identical files recognized correctly
    \item \checkmark{} Cross-precision (3.1 vs 3.14) recognized as equivalent
    \item \checkmark{} All $\binom{15}{2} = 105$ pairwise comparisons pass
    \item \checkmark{} High precision (17 decimal places) handled correctly
\end{itemize}

\subsection{Regression Testing}

The \texttt{make test} target runs the complete test suite:

\begin{lstlisting}[language=bash]
$ make test
Running unit tests...
[==========] 43 tests passed
  - Semantic invariants: 8 tests
  - Sub-LSB boundary: 6 tests
  - Percent threshold: 5 tests
  - Sensitive threshold: 4 tests
  - TL-specific: 12 tests
  - File comparator: 8 tests

All tests passed successfully!
\end{lstlisting}

\subsection{Test File Naming Convention}

Test files follow structured naming:

\begin{verbatim}
test_<scenario>_<variant><file_number>.txt

Examples:
test_2percent_tl1.txt      # 2% threshold, transmission loss, file 1
test_critical1.txt         # Critical difference case, file 1
test_sub_lsb_edge1.txt     # Sub-LSB edge case, file 1
\end{verbatim}

This enables automated test discovery and classification.
