\section{Implementation}
\label{sec:implementation}

\subsection{Code Organization}

The \ubdiff{} implementation is organized into the following components:

\subsubsection{Key Source Files}

\begin{description}
    \item[\texttt{uband\_diff.h}] Threshold definitions, data structures (\texttt{Thresholds}, \texttt{CountStats}, \texttt{DiffStats}, \texttt{Flags})
    \item[\texttt{difference\_analyzer.h/cpp}] Core discrimination logic
    \item[\texttt{file\_comparator.h/cpp}] File parsing, orchestration, summary generation
    \item[\texttt{file\_reader.h/cpp}] Line parsing, precision detection, range data detection
    \item[\texttt{line\_parser.h/cpp}] Low-level numeric parsing
    \item[\texttt{main/uband\_diff.cpp}] Command-line interface
\end{description}

\subsection{Data Structures}

\subsubsection{Runtime Structures}

\begin{lstlisting}[language=C++]
struct ColumnValues {
    double value1, value2;
    int min_dp, max_dp;  // Decimal places
    double range_min, range_max;
};

struct Thresholds {
    double zero;           // Machine epsilon
    double significant;    // User threshold
    double critical;       // Hard failure threshold
    double marginal;       // TL marginal band (110 dB)
    double ignore;         // TL ignore threshold (~138 dB)
    bool significant_is_percent;
    double significant_percent;
};

struct CountStats {
    size_t diff_non_zero;
    size_t diff_trivial;
    size_t diff_non_trivial;
    size_t diff_insignificant;
    size_t diff_significant;
    size_t diff_marginal;
    size_t diff_critical;
    // ... more counters
};

struct DiffStats {
    double max_non_zero;
    double max_non_trivial;
    double max_significant;
    double max_percent_error;
    int max_non_zero_dp;
    int max_non_trivial_dp;
};

struct Flags {
    bool has_critical_diff;
    bool error_found;
    bool column1_is_range_data;
    // ... more flags
};
\end{lstlisting}

\subsection{Analysis Flow}

The comparison pipeline follows this sequence:

\begin{enumerate}
    \item \texttt{FileComparator::compare\_files()} orchestrates file reading
    \item \texttt{FileComparator::process\_line()} parses each line
    \item \texttt{FileComparator::process\_column()} extracts column values
    \item \texttt{FileComparator::process\_difference()} delegates to analyzer
    \item \texttt{DifferenceAnalyzer::process\_difference()} entry point
    \item \texttt{DifferenceAnalyzer::process\_raw\_values()} implements Level 1
    \item \texttt{DifferenceAnalyzer::process\_rounded\_values()} implements Levels 2--5
\end{enumerate}

\subsection{Key Algorithms}

\subsubsection{Decimal Place Detection}

\begin{lstlisting}[language=C++]
int count_decimal_places(const string& str) {
    size_t dot_pos = str.find('.');
    if (dot_pos == string::npos) return 0;

    size_t end_pos = str.find_first_not_of("0123456789", dot_pos + 1);
    if (end_pos == string::npos) end_pos = str.length();

    return static_cast<int>(end_pos - dot_pos - 1);
}
\end{lstlisting}

\subsubsection{Range Data Detection}

\begin{lstlisting}[language=C++]
bool is_first_column_range_data() {
    if (!is_first_column_monotonic()) return false;
    if (!is_first_column_fixed_delta()) return false;

    // Check starting value < 100 (typical for range)
    if (first_column_values[0] >= 100.0) return false;

    return true;
}
\end{lstlisting}

\subsubsection{Rounding to Decimal Places}

\begin{lstlisting}[language=C++]
double round_to_decimals(double value, int dp) {
    if (dp < 0 || dp > MAX_DECIMAL_PLACES) return value;

    double multiplier = pow(10.0, dp);
    return round(value * multiplier) / multiplier;
}
\end{lstlisting}

\subsection{Summary Generation}

Three levels of summary output:

\begin{description}
    \item[\texttt{print\_diff\_like\_summary()}] Level 1 summary (zero vs non-zero)
    \item[\texttt{print\_rounded\_summary()}] Level 2 summary (trivial vs non-trivial)
    \item[\texttt{print\_significant\_summary()}] Levels 3--5 (full discrimination)
\end{description}

Each summary maintains accounting invariants and highlights maximum differences with ANSI color codes (purple underline).
