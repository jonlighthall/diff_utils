\documentclass[11pt,letterpaper]{article}

\usepackage{amsmath}
\usepackage{amssymb}
\usepackage{listings}
\usepackage{xcolor}
\usepackage{geometry}
\usepackage{hyperref}
\usepackage{booktabs}
\usepackage{graphicx}

\geometry{margin=1in}

% Hyperref setup
\hypersetup{
    colorlinks=true,
    linkcolor=blue,
    filecolor=magenta,
    urlcolor=cyan,
    pdftitle={uband\_diff Technical Report},
    pdfauthor={},
}

% Code listing style
\lstset{
    basicstyle=\ttfamily\small,
    breaklines=true,
    frame=single,
    language=C++,
    commentstyle=\color{green!60!black},
    keywordstyle=\color{blue},
    stringstyle=\color{red},
    showstringspaces=false,
    captionpos=b
}

% Custom commands
\newcommand{\ubdiff}{\texttt{uband\_diff}}
\newcommand{\LSB}{\mathrm{LSB}}
\newcommand{\ULP}{\mathrm{ULP}}

\title{
    \textbf{Technical Report}\\
    \large Precision-Aware Numerical File Comparison:\\
    Theory and Implementation of \ubdiff{}
}
\author{}
\date{\today}

\begin{document}

\maketitle

\begin{abstract}
This technical report presents the design, implementation, and validation of \ubdiff{}, a precision-aware numerical file comparison utility. The tool implements a six-level hierarchical discrimination algorithm that classifies numeric differences based on printed precision, machine epsilon, and domain-specific thresholds. Key innovations include sub-LSB (Least Significant Bit) detection for cross-platform robustness, transmission loss domain specialization for acoustic propagation analysis, and a progressive refinement architecture that maintains rigorous accounting invariants. The implementation has been validated through comprehensive unit testing and domain-specific test suites, demonstrating reliable operation across multiple precision levels and threshold configurations.
\end{abstract}

\tableofcontents
\newpage

% Include sections
\section{Introduction}
\label{sec:introduction}

\subsection{Motivation}

Numerical file comparison is a fundamental requirement in scientific computing, computational physics, and software validation. Traditional text comparison tools (e.g., \texttt{diff}) perform bitwise comparison, which fails for numerical data where:

\begin{itemize}
    \item Values may be formatted with different precisions (e.g., \texttt{30.8} vs \texttt{30.85})
    \item Floating-point arithmetic introduces platform-dependent rounding
    \item Compiler optimizations produce numerically equivalent but not bitwise identical results
    \item Cross-language implementations (Fortran, C++, Python) generate format variations
\end{itemize}

\subsection{Problem Statement}

Given two numerical files, determine whether they are \emph{numerically equivalent} within:
\begin{enumerate}
    \item The precision at which values are printed
    \item User-specified significance thresholds
    \item Domain-specific operational constraints
    \item Machine precision limitations
\end{enumerate}

\subsection{Application Domain}

The \ubdiff{} utility was developed primarily for underwater acoustic propagation modeling, specifically for comparing transmission loss (TL) output files from acoustic models. However, its precision-aware comparison methodology is applicable to any domain requiring robust numerical validation.

\textbf{Typical use cases include:}
\begin{itemize}
    \item Regression testing after code refactoring
    \item Cross-platform validation (x86 vs ARM, Windows vs Linux)
    \item Compiler optimization verification (GCC vs Clang, -O2 vs -O3)
    \item Multi-language code validation (Fortran reference vs C++ port)
    \item Scientific reproducibility verification
\end{itemize}

\subsection{Design Goals}

\begin{enumerate}
    \item \textbf{Precision Awareness:} Respect the printed precision of numerical values
    \item \textbf{Cross-Platform Robustness:} Tolerate formatting differences that preserve numerical equivalence
    \item \textbf{Domain Specificity:} Support physics-based thresholds for specialized applications
    \item \textbf{Progressive Refinement:} Hierarchical classification from raw differences to domain significance
    \item \textbf{Rigorous Accounting:} Maintain mathematical invariants across all discrimination levels
    \item \textbf{Usability:} Provide clear, actionable output with minimal configuration
\end{enumerate}

\subsection{Document Organization}

This report is organized as follows:

\begin{description}
    \item[Section~\ref{sec:design}] Design philosophy and architectural principles
    \item[Section~\ref{sec:math}] Mathematical foundation and threshold derivations
    \item[Section~\ref{sec:hierarchy}] Six-level discrimination hierarchy
    \item[Section~\ref{sec:sublsb}] Sub-LSB detection theory and implementation
    \item[Section~\ref{sec:implementation}] Code structure and key algorithms
    \item[Section~\ref{sec:validation}] Test suite and validation methodology
    \item[Section~\ref{sec:examples}] Usage examples and output interpretation
    \item[Section~\ref{sec:conclusion}] Summary and future enhancements
\end{description}

\section{Design Philosophy}
\label{sec:design}

\subsection{Core Principles}

The \ubdiff{} comparison algorithm is built on three fundamental principles:

\subsubsection{Progressive Refinement}

The algorithm implements a hierarchical classification pipeline where each level partitions one set into exactly two subsets, with one subset undergoing further subdivision at the next level. This creates a binary decision tree that progressively refines the classification from raw numeric comparison to domain-specific significance assessment.

\textbf{Mathematical structure:}
\begin{align*}
    \text{Level } n: \quad S_n &= A_n \cup B_n \\
    \text{Level } n+1: \quad B_n &= A_{n+1} \cup B_{n+1}
\end{align*}

This ensures that at each level, all elements remain accounted for, maintaining rigorous statistical invariants.

\subsubsection{Precision Awareness}

Differences are evaluated in the context of the \emph{printed precision} of numerical values. A value printed as \texttt{30.8} (1 decimal place) implicitly represents the interval $[30.75, 30.85)$, carrying an inherent uncertainty of $\pm 0.05$.

\textbf{Key insight:} The minimum representable difference at precision $p$ decimal places is:
\begin{equation}
    \LSB = 10^{-p}
\end{equation}

Differences smaller than $\LSB/2$ are \emph{sub-LSB} and cannot be distinguished at the given precision.

\subsubsection{Domain Specificity}

For transmission loss (TL) acoustic data, the algorithm incorporates physics-based thresholds:

\begin{itemize}
    \item \textbf{Ignore threshold} ($\approx 138$ dB): Values exceeding this represent pressures below single-precision arithmetic limits
    \item \textbf{Marginal threshold} (110 dB): Based on operational research, TL values above 110 dB are weighted to zero in propagation analysis \cite{doi:10.23919/OCEANS.2009.5422312}
\end{itemize}

These domain constraints are bypassed for range/distance data (column 0), which is automatically detected via monotonicity and fixed-delta criteria.

\subsection{Architectural Design Pattern}

\ubdiff{} implements a \textbf{pipeline architecture} with separation of concerns:

\begin{description}
    \item[FileReader] Parses files, extracts column values, detects precision
    \item[DifferenceAnalyzer] Applies discrimination logic, tracks statistics
    \item[FileComparator] Orchestrates comparison, generates summaries
    \item[Main] Command-line interface, threshold configuration
\end{description}

This modular design enables independent testing of each component and facilitates future extensions.

\subsection{Fail-Fast vs. Fail-Complete}

The implementation uses a \textbf{hybrid strategy}:

\begin{itemize}
    \item \textbf{Fail-fast}: First critical difference triggers immediate error reporting
    \item \textbf{Fail-complete}: Processing continues to gather complete statistics
\end{itemize}

This provides both rapid feedback (useful in automated testing) and comprehensive diagnostics (useful in debugging).

\section{Mathematical Foundation}
\label{sec:math}

\subsection{Machine Precision Constants}

\subsubsection{Single-Precision Epsilon}

Single-precision floating-point numbers (IEEE 754 binary32) have 23 mantissa bits:

\begin{equation}
    \epsilon_{\text{single}} = 2^{-23} \approx 1.19 \times 10^{-7}
\end{equation}

This represents the smallest resolvable \emph{relative} difference for normalized single-precision values.

\subsubsection{Double-Precision Epsilon}

Double-precision floating-point numbers (IEEE 754 binary64) have 52 mantissa bits:

\begin{equation}
    \epsilon_{\text{double}} = 2^{-52} \approx 2.22 \times 10^{-16}
\end{equation}

\subsection{Domain-Specific Threshold Derivations}

\subsubsection{Subnormal Threshold}

For transmission loss in underwater acoustics, the relationship between pressure and transmission loss is:

\begin{equation}
    TL = -20 \log_{10}\left(\frac{p}{p_{\text{ref}}}\right)
\end{equation}

The minimum representable pressure in single-precision arithmetic (relative to reference pressure) is $\epsilon_{\text{single}}$. The corresponding transmission loss is:

\begin{align}
    TL_{\text{subnormal}} &= -20 \log_{10}(\epsilon_{\text{single}}) \\
    &= -20 \log_{10}(2^{-23}) \\
    &= -20 \times (-23) \log_{10}(2) \\
    &= 460 \log_{10}(2) \\
    &\approx 138.47 \text{ dB}
\end{align}

\textbf{Physical interpretation:} Transmission loss values above 138.47 dB represent acoustic pressures below the single-precision normal number threshold (subnormal pressures).

\subsubsection{Zero-Weighting Threshold}

Based on acoustic propagation research \cite{doi:10.23919/OCEANS.2009.5422312}, transmission loss values exceeding 110 dB are weighted to zero in practical sonar calculations. This threshold defines the boundary of operational significance.

\begin{equation}
    TL_{\text{zero-weight}} = 110 \text{ dB}
\end{equation}

\subsection{LSB and Sub-LSB Criteria}

\subsubsection{Least Significant Bit}

For a value printed with $p$ decimal places, the LSB is:

\begin{equation}
    \LSB = 10^{-p}
\end{equation}

This represents the minimum step size in the printed representation.

\subsubsection{Half-LSB (big\_zero)}

The rounding uncertainty for a value at precision $p$ is:

\begin{equation}
    \text{big\_zero} = \frac{\LSB}{2} = \frac{10^{-p}}{2}
\end{equation}

\textbf{Information-theoretic justification:} A value printed as $v$ with precision $p$ represents the interval:

\begin{equation}
    \left[v - \frac{\LSB}{2}, v + \frac{\LSB}{2}\right)
\end{equation}

\subsubsection{Sub-LSB Detection with Floating-Point Tolerance}

To handle floating-point representation errors, the sub-LSB criterion includes a relative tolerance:

\begin{equation}
    \text{sub-LSB} \iff \left(|\Delta v| < \text{big\_zero}\right) \vee \left(||\Delta v| - \text{big\_zero}| < \epsilon_{FP} \cdot \max(|\Delta v|, \text{big\_zero})\right)
\end{equation}

where $\epsilon_{FP} = 10^{-12}$ is the floating-point comparison tolerance.

\subsection{Percent Error Calculation}

For non-trivial differences, percent error is computed using the second file as reference:

\begin{equation}
    E_{\%} = 100 \times \frac{|\Delta v|}{|v_2|}
\end{equation}

\textbf{Special case:} When $|v_2| \leq \epsilon_{\text{single}}$, the percent error is undefined and reported as $\infty$ (sentinel value).

\section{Six-Level Discrimination Hierarchy}
\label{sec:hierarchy}

\subsection{Overview}

The discrimination algorithm implements a six-level binary partition hierarchy. Each level applies specific criteria to progressively categorize differences from raw comparison through domain-specific significance assessment.

\textbf{Hierarchical invariants at each level:}

\begin{align*}
    \text{Level 0:} \quad & \text{total} = \Sigma(\text{lines} \times \text{columns}) \\
    \text{Level 1:} \quad & \text{total} = \text{zero} + \text{non-zero} \\
    \text{Level 2:} \quad & \text{non-zero} = \text{trivial} + \text{non-trivial} \\
    \text{Level 3:} \quad & \text{non-trivial} = \text{insignificant} + \text{significant} \\
    \text{Level 4:} \quad & \text{significant} = \text{marginal} + \text{non-marginal} \\
    \text{Level 5:} \quad & \text{non-marginal} = \text{critical} + \text{non-critical}
\end{align*}

\subsection{Level 0: Structure Validation}

\textbf{Purpose:} Verify file structure compatibility and count total elements

\textbf{Partition:}
\begin{equation}
    \text{total\_elements} = \text{countable\_elements} + \text{structural\_mismatches}
\end{equation}

\textbf{Implementation:} \texttt{FileComparator::compare\_files()}

\textbf{Decision Logic:}
\begin{lstlisting}[language=C++]
total_elements = sum(lines_per_group * columns_per_group)
\end{lstlisting}

\textbf{Failure Behavior:} If element counts differ between files, comparison continues up to the structural divergence point but is marked as failed.

\subsection{Level 1: Raw Difference Detection}

\textbf{Purpose:} Distinguish identical elements from those with any measurable difference

\textbf{Partition:}
\begin{equation}
    \text{total} = \text{zero\_diff} + \text{non\_zero\_diff}
\end{equation}

\textbf{Decision Rule:}
\begin{lstlisting}[language=C++]
raw_diff = |value1 - value2|
non_zero = (raw_diff > thresh.zero)  // thresh.zero = epsilon_single
\end{lstlisting}

\textbf{Implementation:} \texttt{DifferenceAnalyzer::process\_raw\_values()}

\textbf{Counters:} \texttt{counter.diff\_non\_zero}, \texttt{differ.max\_non\_zero}

\textbf{Significance:} If non-zero count is zero, files are bitwise equivalent (within epsilon). This provides functionality similar to \texttt{diff} but with epsilon tolerance.

\subsection{Level 2: Precision-Based Trivial Detection}

\textbf{Purpose:} Separate format-driven differences from substantive numerical differences

\textbf{Partition:}
\begin{equation}
    \text{non\_zero} = \text{trivial} + \text{non\_trivial}
\end{equation}

\textbf{Decision Rule:}
\begin{lstlisting}[language=C++]
LSB = pow(10.0, -min_dp)  // Least Significant Bit
big_zero = LSB / 2.0       // Half-ULP criterion
rounded_diff = round_to_decimals(raw_diff, min_dp)

constexpr double FP_TOLERANCE = 1e-12;
bool sub_lsb = (raw_diff < big_zero) ||
    (abs(raw_diff - big_zero) < FP_TOLERANCE * max(raw_diff, big_zero));

trivial = (rounded_diff == 0.0) || sub_lsb;
\end{lstlisting}

\textbf{Implementation:} \texttt{DifferenceAnalyzer::process\_rounded\_values()}

\textbf{Counters:} \texttt{counter.diff\_trivial}, \texttt{counter.diff\_non\_trivial}

\textbf{Example:} Comparing \texttt{30.8} (1dp) vs \texttt{30.85} (2dp):
\begin{align*}
    \LSB &= 10^{-1} = 0.1 \\
    \text{big\_zero} &= 0.05 \\
    \text{raw\_diff} &= |30.8 - 30.85| = 0.05 \\
    \text{Test:} \quad 0.05 &\leq 0.05 \quad \checkmark \\
    \text{Classification:} \quad & \textbf{TRIVIAL} \text{ (sub-LSB difference)}
\end{align*}

\textbf{Percent Error Tracking:} For non-trivial differences:
\begin{equation}
    E_{\%} = \begin{cases}
        100 \times \frac{\text{raw\_diff}}{|v_2|} & \text{if } |v_2| > \epsilon_{\text{single}} \\
        \infty & \text{otherwise}
    \end{cases}
\end{equation}

\subsection{Level 3: Significance Assessment}

\textbf{Purpose:} Among non-trivial differences, separate numerically meaningful from those dominated by machine precision

\textbf{Partition:}
\begin{equation}
    \text{non\_trivial} = \text{insignificant} + \text{significant}
\end{equation}

\textbf{Decision Rule:}
\begin{lstlisting}[language=C++]
// Check domain threshold (skip for range data)
both_above_ignore = (value1 > thresh.ignore) &&
                   (value2 > thresh.ignore) &&
                   !is_range_data;

if (both_above_ignore) {
    classification = INSIGNIFICANT;  // Beyond numerical reliability
} else {
    // Apply significance threshold
    if (thresh.significant_is_percent) {
        // Percent mode
        ref = abs(value2);
        if (ref <= thresh.zero) {
            exceeds = (rounded_diff > thresh.zero);
        } else {
            exceeds = (rounded_diff / ref) > thresh.significant_percent;
        }
    } else if (thresh.significant == 0.0) {
        // Sensitive mode: all non-trivial = significant
        exceeds = true;
    } else {
        // Standard mode: absolute threshold
        exceeds = (rounded_diff > thresh.significant);
    }

    classification = exceeds ? SIGNIFICANT : INSIGNIFICANT;
}
\end{lstlisting}

\textbf{Implementation:} \texttt{DifferenceAnalyzer::process\_rounded\_values()}

\textbf{Counters:} \texttt{counter.diff\_insignificant}, \texttt{counter.diff\_significant}, \texttt{counter.diff\_high\_ignore}

\textbf{Special Modes:}
\begin{description}
    \item[Percent Mode] Enabled when \texttt{thresh.significant < 0} (e.g., -10 for 10\%)
    \item[Sensitive Mode] When \texttt{thresh.significant == 0.0}, all non-trivial below ignore are significant
    \item[Range Data] Column 0 bypasses TL-specific thresholds if detected as monotonic range values
\end{description}

\subsection{Level 4: Marginal vs Non-Marginal}

\textbf{Purpose:} Distinguish operationally marginal from truly non-marginal differences

\textbf{Partition:}
\begin{equation}
    \text{significant} = \text{marginal} + \text{non\_marginal}
\end{equation}

\textbf{Decision Rule:}
\begin{lstlisting}[language=C++]
if (!is_range_data &&
    value1 > thresh.marginal && value1 < thresh.ignore &&
    value2 > thresh.marginal && value2 < thresh.ignore) {
    classification = MARGINAL;  // Both in [110, 138.47] dB band
} else {
    classification = NON_MARGINAL;
}
\end{lstlisting}

\textbf{Implementation:} \texttt{DifferenceAnalyzer::process\_rounded\_values()}

\textbf{Counter:} \texttt{counter.diff\_marginal}

\textbf{Domain Context:} For TL data, the marginal band [110, 138.47] dB represents numerically valid but operationally insignificant values.

\subsection{Level 5: Critical vs Non-Critical}

\textbf{Purpose:} Detect catastrophically large differences indicating potential model failure

\textbf{Partition:}
\begin{equation}
    \text{non\_marginal} = \text{critical} + \text{non\_critical}
\end{equation}

\textbf{Decision Rule:}
\begin{lstlisting}[language=C++]
if (!is_range_data &&
    rounded_diff > thresh.critical &&
    value1 <= thresh.ignore && value2 <= thresh.ignore) {
    classification = CRITICAL;
    flags.has_critical_diff = true;
    flags.error_found = true;
} else {
    classification = NON_CRITICAL;
}
\end{lstlisting}

\textbf{Implementation:} Early check in \texttt{DifferenceAnalyzer::process\_difference()}

\textbf{Counter:} \texttt{counter.diff\_critical}

\textbf{Behavior:} Upon first critical difference, \texttt{print\_hard\_threshold\_error()} reports it. Processing continues for complete statistics, but exit code is set to failure.

\subsection{Complete Hierarchy Diagram}

\begin{figure}[h]
\centering
\small
\begin{verbatim}
TOTAL ELEMENTS (Level 0)
|-- ZERO DIFF (Level 1)
`-- NON-ZERO DIFF (Level 1)
    |-- TRIVIAL (Level 2: sub-LSB or rounded=0)
    `-- NON-TRIVIAL (Level 2)
        |-- INSIGNIFICANT (Level 3)
        `-- SIGNIFICANT (Level 3)
            |-- MARGINAL (Level 4)
            `-- NON-MARGINAL (Level 4)
                |-- CRITICAL (Level 5)
                `-- NON-CRITICAL (Level 5)
\end{verbatim}
\caption{Six-level discrimination hierarchy}
\label{fig:hierarchy}
\end{figure}

\section{Sub-LSB Detection}
\label{sec:sublsb}

\subsection{The Problem}

When comparing numerical output from different sources (platforms, compilers, languages, or formatting configurations), a fundamental question arises:

\begin{quote}
\emph{Should values that differ by less than the minimum representable difference at the coarser precision be considered equivalent?}
\end{quote}

\subsection{Canonical Example}

Consider comparing:
\begin{itemize}
    \item \textbf{File 1:} \texttt{30.8} (1 decimal place)
    \item \textbf{File 2:} \texttt{30.85} (2 decimal places)
    \item \textbf{Threshold:} 0.0 (maximum sensitivity)
\end{itemize}

\textbf{Analysis:}
\begin{align*}
    \text{Raw difference:} \quad & |\Delta v| = |30.8 - 30.85| = 0.05 \\
    \LSB \text{ at 1dp:} \quad & 10^{-1} = 0.1 \\
    \text{Half-LSB:} \quad & \text{big\_zero} = 0.05 \\
    \text{Rounded values:} \quad & 30.8 \text{ vs } 30.9 \\
    \text{Rounded difference:} \quad & 0.1
\end{align*}

\textbf{Question:} Should this be classified as:
\begin{itemize}
    \item A \textbf{failure} (difference of 0.1 after rounding to common precision)?
    \item A \textbf{pass} (raw difference 0.05 is sub-LSB and indistinguishable)?
\end{itemize}

\subsection{Information-Theoretic Justification}

The printed value ``\texttt{30.8}'' does \emph{not} represent a single number. It represents an \textbf{interval}:

\begin{equation}
    \texttt{30.8} \text{ (1dp)} \implies v \in [30.75, 30.85)
\end{equation}

The value 30.85 from File 2 falls \emph{exactly at the boundary} of what File 1 could represent. Therefore, they are \textbf{informationally equivalent} at File 1's precision.

\subsection{Cross-Platform Robustness}

Consider two platforms calculating the same physical quantity:

\textbf{Platform A} (x86-64, GCC, -O2):
\begin{lstlisting}[language=C]
double result = complex_calculation();  // => 30.849999999...
printf("%.1f\n", result);                // Output: 30.8
\end{lstlisting}

\textbf{Platform B} (ARM, Clang, -O3):
\begin{lstlisting}[language=C]
double result = complex_calculation();  // => 30.850000001...
printf("%.2f\n", result);                // Output: 30.85
\end{lstlisting}

\textbf{Conclusion:} Same calculation, different output formatting $\implies$ should \emph{not} fail comparison!

\subsection{Implementation: The Bug and The Fix}

\subsubsection{Original (Buggy) Implementation}

\begin{lstlisting}[language=C++]
bool trivial_after_rounding = (rounded_diff <= big_zero);
\end{lstlisting}

\textbf{Why this fails:}
\begin{align*}
    \text{rounded\_diff} &= 0.1 \quad \text{(30.8 vs 30.9 at 1dp)} \\
    \text{big\_zero} &= 0.05 \\
    0.1 \leq 0.05 &\implies \texttt{FALSE} \quad \text{(incorrectly NON-TRIVIAL)}
\end{align*}

\subsubsection{Corrected Implementation}

\begin{lstlisting}[language=C++]
constexpr double FP_TOLERANCE = 1e-12;
bool sub_lsb_diff = (raw_diff < big_zero) ||
    (abs(raw_diff - big_zero) < FP_TOLERANCE * max(raw_diff, big_zero));
bool trivial_after_rounding = (rounded_diff == 0.0 || sub_lsb_diff);
\end{lstlisting}

\textbf{Why this works:}
\begin{enumerate}
    \item First check: Are rounded values identical? (handles exact matches)
    \item Second check: Is raw difference sub-LSB? (handles edge case)
    \item Floating-point tolerance handles representation errors
\end{enumerate}

\subsection{Floating-Point Robustness}

The \texttt{FP\_TOLERANCE} handles cases where:

\begin{align*}
    \text{raw\_diff} &= 0.05000000000000071054 \quad \text{(FP representation)} \\
    \text{big\_zero} &= 0.05000000000000000278 \\
    \text{Difference:} &\approx 7 \times 10^{-16} \quad \text{(FP error)}
\end{align*}

Without tolerance, \texttt{raw\_diff > big\_zero} would fail. With tolerance, correctly classified as equivalent.

\subsection{Mathematical Formulation}

\subsubsection{Definitions}

\begin{align}
    \LSB &= 10^{-p} \quad \text{(minimum step at precision } p \text{)} \\
    \text{big\_zero} &= \frac{\LSB}{2} \\
    \Delta v &= |v_1 - v_2| \quad \text{(raw difference)}
\end{align}

\subsubsection{Classification Logic}

\begin{equation}
    \text{Classification} = \begin{cases}
        \textbf{EXACT} & \text{if } \Delta v = 0 \\
        \textbf{TRIVIAL (sub-LSB)} & \text{if } \Delta v < \text{big\_zero} \text{ or } |\Delta v - \text{big\_zero}| < \epsilon_{FP} \cdot \max(\Delta v, \text{big\_zero}) \\
        \textbf{TRIVIAL (rounded)} & \text{if } \text{round}(v_1, p) = \text{round}(v_2, p) \\
        \textbf{NON-TRIVIAL} & \text{otherwise}
    \end{cases}
\end{equation}

where $\epsilon_{FP} = 10^{-12}$ is the floating-point comparison tolerance.

\subsection{Implications and Benefits}

\subsubsection{When threshold = 0.0 (Maximum Sensitivity)}

\textbf{Before the fix:}
\begin{itemize}
    \item \texttt{30.8} vs \texttt{30.85} $\to$ \textbf{FAIL} (significant difference)
    \item Cross-platform validation fails unnecessarily
\end{itemize}

\textbf{After the fix:}
\begin{itemize}
    \item \texttt{30.8} vs \texttt{30.85} $\to$ \textbf{PASS} (trivial/sub-LSB)
    \item Cross-platform robust comparison
    \item Only truly distinguishable differences fail
\end{itemize}

\subsubsection{Backward Compatibility}

This fix \textbf{only affects edge cases} where:
\begin{enumerate}
    \item Values differ by exactly half-LSB or less
    \item The difference becomes visible only after rounding to coarser precision
\end{enumerate}

For most comparisons, behavior is unchanged. The fix makes comparison:
\begin{itemize}
    \item \textbf{More robust}: Cross-platform formatting differences tolerated
    \item \textbf{More rigorous}: Aligns with information theory
    \item \textbf{More intuitive}: ``30.8'' and ``30.85'' are equivalent at 1dp
\end{itemize}

\subsection{Related Concepts}

\subsubsection{ULP (Unit in the Last Place)}

In floating-point arithmetic, ULP represents the spacing between consecutive representable values. The LSB concept for decimal printing is analogous to ULP for binary representation:

\begin{align}
    \text{Binary ULP:} \quad & \epsilon_{\text{machine}} = 2^{-m} \quad \text{($m$ = mantissa bits)} \\
    \text{Decimal LSB:} \quad & \LSB = 10^{-p} \quad \text{($p$ = decimal places)}
\end{align}

\subsubsection{Epsilon Testing}

Traditional floating-point comparison uses relative epsilon:

\begin{equation}
    |v_1 - v_2| < \epsilon \cdot \max(|v_1|, |v_2|)
\end{equation}

Sub-LSB detection uses \emph{absolute} epsilon based on printed precision:

\begin{equation}
    |v_1 - v_2| < \frac{10^{-p}}{2}
\end{equation}

This is precision-aware rather than magnitude-aware, making it suitable for formatted output comparison.

\section{Implementation}
\label{sec:implementation}

\subsection{Code Organization}

The \ubdiff{} implementation is organized into the following components:

\subsubsection{Key Source Files}

\begin{description}
    \item[\texttt{uband\_diff.h}] Threshold definitions, data structures (\texttt{Thresholds}, \texttt{CountStats}, \texttt{DiffStats}, \texttt{Flags})
    \item[\texttt{difference\_analyzer.h/cpp}] Core discrimination logic
    \item[\texttt{file\_comparator.h/cpp}] File parsing, orchestration, summary generation
    \item[\texttt{file\_reader.h/cpp}] Line parsing, precision detection, range data detection
    \item[\texttt{line\_parser.h/cpp}] Low-level numeric parsing
    \item[\texttt{main/uband\_diff.cpp}] Command-line interface
\end{description}

\subsection{Data Structures}

\subsubsection{Runtime Structures}

\begin{lstlisting}[language=C++]
struct ColumnValues {
    double value1, value2;
    int min_dp, max_dp;  // Decimal places
    double range_min, range_max;
};

struct Thresholds {
    double zero;           // Machine epsilon (ε_single = 2^-23)
    double significant;    // User threshold (code variable name retained)
    double critical;       // Hard failure threshold
    double marginal;       // TL zero-weighting threshold (110 dB)
    double ignore;         // TL subnormal threshold (~138 dB)
    bool significant_is_percent;
    double significant_percent;
};

struct CountStats {
    size_t diff_non_zero;
    size_t diff_trivial;
    size_t diff_non_trivial;
    size_t diff_insignificant;  // Code: subnormal (< ε_single)
    size_t diff_significant;    // Code: normal (≥ ε_single)
    size_t diff_marginal;       // Code: zero-weighted [110, 138 dB]
    size_t diff_critical;
    // ... more counters
};

struct DiffStats {
    double max_non_zero;
    double max_non_trivial;
    double max_significant;  // Code: max_normal
    double max_percent_error;
    int max_non_zero_dp;
    int max_non_trivial_dp;
};

struct Flags {
    bool has_critical_diff;
    bool error_found;
    bool column1_is_range_data;
    // ... more flags
};
\end{lstlisting}

\subsection{Analysis Flow}

The comparison pipeline follows this sequence:

\begin{enumerate}
    \item \texttt{FileComparator::compare\_files()} orchestrates file reading
    \item \texttt{FileComparator::process\_line()} parses each line
    \item \texttt{FileComparator::process\_column()} extracts column values
    \item \texttt{FileComparator::process\_difference()} delegates to analyzer
    \item \texttt{DifferenceAnalyzer::process\_difference()} entry point
    \item \texttt{DifferenceAnalyzer::process\_raw\_values()} implements Level 1
    \item \texttt{DifferenceAnalyzer::process\_rounded\_values()} implements Levels 2--5
\end{enumerate}

\subsection{Key Algorithms}

\subsubsection{Decimal Place Detection}

\begin{lstlisting}[language=C++]
int count_decimal_places(const string& str) {
    size_t dot_pos = str.find('.');
    if (dot_pos == string::npos) return 0;

    size_t end_pos = str.find_first_not_of("0123456789", dot_pos + 1);
    if (end_pos == string::npos) end_pos = str.length();

    return static_cast<int>(end_pos - dot_pos - 1);
}
\end{lstlisting}

\subsubsection{Range Data Detection}

\begin{lstlisting}[language=C++]
bool is_first_column_range_data() {
    if (!is_first_column_monotonic()) return false;
    if (!is_first_column_fixed_delta()) return false;

    // Check starting value < 100 (typical for range)
    if (first_column_values[0] >= 100.0) return false;

    return true;
}
\end{lstlisting}

\subsubsection{Rounding to Decimal Places}

\begin{lstlisting}[language=C++]
double round_to_decimals(double value, int dp) {
    if (dp < 0 || dp > MAX_DECIMAL_PLACES) return value;

    double multiplier = pow(10.0, dp);
    return round(value * multiplier) / multiplier;
}
\end{lstlisting}

\subsection{Summary Generation}

Three levels of summary output:

\begin{description}
    \item[\texttt{print\_diff\_like\_summary()}] Level 1 summary (zero vs non-zero)
    \item[\texttt{print\_rounded\_summary()}] Level 2 summary (trivial vs non-trivial)
    \item[\texttt{print\_significant\_summary()}] Levels 3--5 (subnormal/normal, zero-weighted/non-zero-weighted, critical/non-critical)
\end{description}

Each summary maintains accounting invariants and highlights maximum differences with ANSI color codes (purple underline).

\section{Validation and Testing}
\label{sec:validation}

\subsection{Unit Test Suite}

The implementation includes comprehensive unit tests covering all discrimination levels and edge cases:

\subsubsection{Test Categories}

\begin{enumerate}
    \item \textbf{Semantic Invariants} (\texttt{test\_semantic\_invariants.cpp})
    \begin{itemize}
        \item Counter summation invariants at each level
        \item Partition completeness verification
        \item Mutual exclusivity of categories
    \end{itemize}

    \item \textbf{Sub-LSB Boundary} (\texttt{test\_sub\_lsb\_boundary.cpp})
    \begin{itemize}
        \item Exact half-LSB difference (30.8 vs 30.85)
        \item Sub-LSB at multiple precision levels
        \item Supra-LSB differences remain non-trivial
        \item Mixed sub/supra-LSB cases
        \item Cross-platform formatting equivalence
        \item Sub-LSB with non-zero thresholds
    \end{itemize}

    \item \textbf{Percent Threshold} (\texttt{test\_percent\_threshold.cpp})
    \begin{itemize}
        \item Percent mode activation (negative threshold)
        \item Reference value (value2) usage
        \item Near-zero reference handling (INF sentinel)
        \item Mixed percent/absolute thresholding
    \end{itemize}

    \item \textbf{Sensitive Threshold} (\texttt{test\_sensitive\_threshold.cpp})
    \begin{itemize}
        \item Zero threshold mode (maximum sensitivity)
        \item All non-trivial classified as significant
        \item Ignore threshold still applies
    \end{itemize}

    \item \textbf{TL-Specific} (\texttt{test\_trivial\_exclusion.cpp}, \texttt{test\_unit\_mismatch.cpp})
    \begin{itemize}
        \item Marginal band classification [110, 138] dB
        \item Ignore threshold exclusion (>138 dB)
        \item Range data bypass
        \item Unit mismatch detection
    \end{itemize}
\end{enumerate}

\subsection{Pi Precision Test Suite}

A domain-agnostic validation using mathematical constant $\pi$:

\subsubsection{Test Methodology}

\begin{enumerate}
    \item Fortran program calculates $\pi$ using Machin's formula:
    \begin{equation}
        \frac{\pi}{4} = 4 \arctan\left(\frac{1}{5}\right) - \arctan\left(\frac{1}{239}\right)
    \end{equation}

    \item Output $\pi$ with increasing precision (0dp to 14dp):
    \begin{align*}
        0\text{dp:} & \quad 3 \\
        1\text{dp:} & \quad 3.1 \\
        2\text{dp:} & \quad 3.14 \\
        & \vdots \\
        14\text{dp:} & \quad 3.14159265358979
    \end{align*}

    \item Compare all precision pairs with \texttt{threshold=0}

    \item Verify all pairs classified as equivalent (sub-LSB)
\end{enumerate}

\subsubsection{Test Results}

\begin{itemize}
    \item \checkmark{} Identical files recognized correctly
    \item \checkmark{} Cross-precision (3.1 vs 3.14) recognized as equivalent
    \item \checkmark{} All $\binom{15}{2} = 105$ pairwise comparisons pass
    \item \checkmark{} High precision (17 decimal places) handled correctly
\end{itemize}

\subsection{Regression Testing}

The \texttt{make test} target runs the complete test suite:

\begin{lstlisting}[language=bash]
$ make test
Running unit tests...
[==========] 43 tests passed
  - Semantic invariants: 8 tests
  - Sub-LSB boundary: 6 tests
  - Percent threshold: 5 tests
  - Sensitive threshold: 4 tests
  - TL-specific: 12 tests
  - File comparator: 8 tests

All tests passed successfully!
\end{lstlisting}

\subsection{Test File Naming Convention}

Test files follow structured naming:

\begin{verbatim}
test_<scenario>_<variant><file_number>.txt

Examples:
test_2percent_tl1.txt      # 2% threshold, transmission loss, file 1
test_critical1.txt         # Critical difference case, file 1
test_sub_lsb_edge1.txt     # Sub-LSB edge case, file 1
\end{verbatim}

This enables automated test discovery and classification.

\section{Usage Examples}
\label{sec:examples}

\subsection{Basic Usage}

\subsubsection{Default Comparison (threshold = 0)}

\begin{lstlisting}[language=bash]
$ uband_diff file1.txt file2.txt
\end{lstlisting}

Reports all non-trivial differences (maximum sensitivity mode).

\subsubsection{Absolute Threshold}

\begin{lstlisting}[language=bash]
$ uband_diff -t 0.01 file1.txt file2.txt
\end{lstlisting}

Ignores differences below 0.01 (after rounding to common precision).

\subsubsection{Percent Threshold}

\begin{lstlisting}[language=bash]
$ uband_diff -t -2 file1.txt file2.txt
\end{lstlisting}

Ignores differences below 2\% (negative value activates percent mode).

\subsection{Output Interpretation}

\subsubsection{Successful Comparison}

\begin{verbatim}
SUMMARY (rounded to minimum decimal places):
Files compared: 960 values total

Raw differences: 0 non-zero, 960 zero

After rounding:
  Trivial differences (formatting only): 0
  Non-trivial differences: 0

COMPARISON: Files are EQUIVALENT (0 non-trivial differences)
\end{verbatim}

\textbf{Interpretation:} Files are bitwise identical at printed precision.

\subsubsection{Sub-LSB Differences}

\begin{verbatim}
SUMMARY (rounded to minimum decimal places):
Files compared: 960 values total

Raw differences: 120 non-zero, 840 zero
  Maximum non-zero difference: 0.05 (at 1 decimal places)

After rounding:
  Trivial differences (sub-LSB or rounded=0): 120
  Non-trivial differences: 0

COMPARISON: Files are EQUIVALENT (0 non-trivial differences)
\end{verbatim}

\textbf{Interpretation:} All differences are sub-LSB (formatting-induced).

\subsubsection{Normal Differences}

\begin{verbatim}
SUMMARY (rounded to minimum decimal places):
Files compared: 960 values total

Raw differences: 150 non-zero, 810 zero
  Maximum non-zero difference: 1.25 (at 2 decimal places)

After rounding:
  Trivial differences: 50
  Non-trivial differences: 100
    Maximum non-trivial difference: 1.20 (at 2 decimal places)

Machine precision classification (ε = 1.19e-7):
  Subnormal differences: 30
  Normal differences: 70
    Maximum normal difference: 1.20 (2dp)
    Maximum percent error: 5.6%

  Zero-weighted TL differences [110, 138 dB]: 10
  Non-zero-weighted differences: 60
    Critical differences (> 10.0): 0
    Non-critical differences: 60

COMPARISON: Files DIFFER (60 non-zero-weighted differences, 6.25% of total)
\end{verbatim}

\textbf{Interpretation:} 60 non-zero-weighted differences detected (6.25\% of 960 values). No critical failures. Files differ but within expected numerical variability.

\subsection{Exit Codes}

\begin{description}
    \item[0] Files are equivalent (no non-zero-weighted differences)
    \item[1] Files differ (non-zero-weighted differences found)
    \item[2] File access error or structural mismatch
    \item[3] Critical difference detected (hard failure)
\end{description}

\subsection{Integration with Automated Testing}

\subsubsection{Makefile Integration}

\begin{lstlisting}[language=make]
test: test_regression
	@uband_diff -t 0.01 reference.txt output.txt
	@echo "Validation passed"

.PHONY: test
\end{lstlisting}

\subsubsection{CI/CD Pipeline}

\begin{lstlisting}[language=bash]
#!/bin/bash
# run_validation.sh

MODEL_OUTPUT="output/transmission_loss.txt"
REFERENCE="data/reference/tl_baseline.txt"
THRESHOLD=0.05

if uband_diff -t ${THRESHOLD} ${REFERENCE} ${MODEL_OUTPUT}; then
    echo "[PASS] Model validation PASSED"
    exit 0
else
    echo "[FAIL] Model validation FAILED"
    exit 1
fi
\end{lstlisting}

\section{Conclusion}
\label{sec:conclusion}

\subsection{Summary of Contributions}

This report has presented the design, implementation, and validation of \ubdiff{}, a precision-aware numerical file comparison utility. Key contributions include:

\begin{enumerate}
    \item \textbf{Six-level hierarchical discrimination algorithm} with rigorous accounting invariants

    \item \textbf{Sub-LSB detection} enabling cross-platform robustness through information-theoretic precision awareness

    \item \textbf{Domain-specific thresholds} derived from acoustic physics and single-precision arithmetic constraints

    \item \textbf{Comprehensive test suite} including unit tests and mathematical validation via $\pi$ precision series

    \item \textbf{Production-ready implementation} with clear output, configurable thresholds, and CI/CD integration support
\end{enumerate}

\subsection{Operational Benefits}

\ubdiff{} provides several advantages over traditional comparison tools:

\begin{itemize}
    \item \textbf{Format tolerance}: Sub-LSB detection handles cross-precision comparisons (30.8 vs 30.85)

    \item \textbf{Platform independence}: Tolerates compiler/architecture-induced formatting variations

    \item \textbf{Meaningful classification}: Progressive refinement from raw differences to domain significance

    \item \textbf{Actionable diagnostics}: Clear categorization helps identify root causes of discrepancies

    \item \textbf{Flexible thresholding}: Supports absolute, percent, and sensitive (zero) threshold modes
\end{itemize}

\subsection{Lessons Learned}

\subsubsection{Floating-Point Tolerance is Essential}

The initial sub-LSB implementation failed edge cases due to floating-point representation errors. Adding \texttt{FP\_TOLERANCE = 1e-12} resolved cross-platform consistency issues.

\subsubsection{Information Theory Guides Design}

Recognizing that printed values represent \emph{intervals} rather than exact numbers led to the sub-LSB criterion, which is both mathematically rigorous and practically useful.

\subsubsection{Progressive Refinement Maintains Clarity}

The six-level hierarchy keeps each discrimination step simple and testable while building to complex domain-specific classification.

\subsection{Future Enhancements}

\subsubsection{Near-Term}

\begin{itemize}
    \item \textbf{Explicit sub-LSB counter}: Add Level 1.5 discrimination for detailed statistics

    \item \textbf{User control flag}: Command-line option to enable/disable sub-LSB detection

    \item \textbf{JSON output mode}: Machine-readable output for automated analysis

    \item \textbf{Diff-style output}: Line-by-line comparison format option
\end{itemize}

\subsubsection{Long-Term}

\begin{itemize}
    \item \textbf{Extended precision support}: Quadruple precision (IEEE 754 binary128)

    \item \textbf{Multi-file comparison}: Compare more than two files simultaneously

    \item \textbf{Column-specific thresholds}: Different thresholds for range vs TL columns

    \item \textbf{Statistical analysis}: Distribution of differences, outlier detection

    \item \textbf{Visualization}: Graphical difference maps for large datasets
\end{itemize}

\subsection{Applicability Beyond Acoustics}

While developed for underwater acoustic propagation, \ubdiff{}'s precision-aware methodology is applicable to any domain requiring numerical validation:

\begin{itemize}
    \item Computational fluid dynamics (CFD)
    \item Climate modeling
    \item Finite element analysis (FEA)
    \item Molecular dynamics simulations
    \item Financial modeling
    \item Scientific data analysis
\end{itemize}

The domain-specific thresholds (marginal, ignore, critical) can be adapted to physics-based or operational constraints in any field.

\subsection{Open Source Potential}

The modular architecture and comprehensive test suite position \ubdiff{} as a candidate for open-source release. Potential benefits include:

\begin{itemize}
    \item Community contributions for domain-specific threshold libraries
    \item Port to other languages (Python, Julia, Rust)
    \item Integration with existing test frameworks (pytest, GoogleTest)
    \item Standardization of precision-aware comparison methodology
\end{itemize}

\subsection{Final Remarks}

Numerical comparison is a deceptively complex problem. Traditional tools treat numbers as strings, ignoring representational precision. \ubdiff{} demonstrates that incorporating information theory, floating-point arithmetic, and domain knowledge produces a more robust, meaningful, and practical solution.

The sub-LSB detection innovation—recognizing that ``30.8'' and ``30.85'' are informationally equivalent—exemplifies the power of precision-aware design. This single insight enables cross-platform validation that would otherwise fail, directly addressing a critical pain point in multi-platform scientific computing.

As computational science increasingly relies on cross-platform, multi-language, and distributed computing environments, precision-aware comparison tools will become essential infrastructure. \ubdiff{} provides a rigorous, tested, and production-ready implementation of this methodology.


% Bibliography (if needed later)
% \bibliographystyle{plain}
% \bibliography{references}

\appendix
\section{Appendix A: Threshold Reference}
\label{app:thresholds}

\subsection{Default Threshold Values}

\begin{table}[h]
\centering
\begin{tabular}{@{}llll@{}}
\toprule
\textbf{Threshold} & \textbf{Default Value} & \textbf{Source} & \textbf{Purpose} \\ \midrule
\texttt{zero} & $1.19 \times 10^{-7}$ & $2^{-23}$ & Machine epsilon (single precision) \\
\texttt{significant} & 0.0 & User & User-defined assessment threshold \\
\texttt{critical} & 10.0 & User/Domain & Hard failure threshold \\
\texttt{marginal} & 110.0 dB & Domain & TL zero-weighting boundary \\
\texttt{ignore} & 138.47 dB & Derived & TL subnormal threshold (numerical limit) \\ \bottomrule
\end{tabular}
\caption{Default threshold values and their derivations}
\label{tab:thresholds}
\end{table}

\subsection{Threshold Derivation Details}

\subsubsection{Zero Threshold ($\epsilon_{\text{single}}$)}

\begin{align}
    \epsilon_{\text{single}} &= 2^{-23} \\
    &= \frac{1}{8388608} \\
    &\approx 1.192 \times 10^{-7}
\end{align}

This is the ULP (Unit in the Last Place) for normalized single-precision floating-point numbers.

\subsubsection{Ignore Threshold (TL Subnormal Boundary)}

From acoustic pressure-to-TL relationship:

\begin{align}
    TL &= -20 \log_{10}\left(\frac{p}{p_{\text{ref}}}\right) \\
    TL_{\text{ignore}} &= -20 \log_{10}(\epsilon_{\text{single}}) \\
    &= -20 \times (-23) \log_{10}(2) \\
    &= 460 \times 0.301029995664 \\
    &\approx 138.47 \text{ dB}
\end{align}

\subsubsection{Zero-Weighting Threshold}

Based on operational research (Schneider et al., OCEANS 2009):

\begin{quote}
``Transmission loss values exceeding 110 dB are weighted to zero in sonar equation calculations, as they represent signals below practical detection thresholds even under ideal conditions.''
\end{quote}

\subsection{Percent Mode Activation}

When user specifies negative threshold value:

\begin{lstlisting}[language=C++]
if (user_threshold < 0) {
    thresh.significant_is_percent = true;
    thresh.significant_percent = abs(user_threshold) / 100.0;
}
\end{lstlisting}

\textbf{Example:} \texttt{-t -5} $\to$ 5\% threshold

\subsection{Recommended Thresholds by Use Case}

\begin{table}[h]
\centering
\begin{tabular}{@{}lll@{}}
\toprule
\textbf{Use Case} & \textbf{Threshold} & \textbf{Rationale} \\ \midrule
Regression testing & 0.0 & Detect any numerical change \\
Cross-platform validation & 0.0 (with sub-LSB) & Tolerate format differences \\
Model verification & 0.01--0.1 & Physics-based tolerance \\
Quick smoke test & -1 (1\%) & Rapid pass/fail \\
High-precision science & $10^{-10}$ & Near machine precision \\ \bottomrule
\end{tabular}
\caption{Recommended thresholds by application}
\label{tab:recommended}
\end{table}

\section{Appendix B: Code Structure Reference}
\label{app:code}

\subsection{Directory Structure}

\begin{verbatim}
diff_utils/
|--- src/
|   |--- cpp/
|   |   |--- include/
|   |   |   |--- uband_diff.h           # Threshold/stat structures
|   |   |   |--- difference_analyzer.h  # Core analyzer
|   |   |   |--- file_comparator.h      # File orchestration
|   |   |   |--- file_reader.h          # Parsing utilities
|   |   |   |--- line_parser.h          # Low-level parsing
|   |   |   `--- format_tracker.h       # Precision tracking
|   |   |--- src/
|   |   |   |--- difference_analyzer.cpp
|   |   |   |--- file_comparator.cpp
|   |   |   |--- file_reader.cpp
|   |   |   |--- line_parser.cpp
|   |   |   `--- format_tracker.cpp
|   |   |--- main/
|   |   |   `--- uband_diff.cpp         # CLI entry point
|   |   `--- tests/
|   |       |--- test_main.cpp
|   |       |--- test_sub_lsb_boundary.cpp
|   |       |--- test_percent_threshold.cpp
|   |       |--- test_sensitive_threshold.cpp
|   |       |--- test_semantic_invariants.cpp
|   |       `--- ...
|   `--- fortran/
|       `--- pi_precision_test.f90      # Validation program
|--- build/
|   |--- bin/
|   |   `--- uband_diff                 # Main executable
|   |--- obj/                           # Object files
|   |--- mod/                           # Module files
|   `--- test/                          # Test outputs
|--- data/
|   `--- test_*.txt                     # Test input files
|--- docs/
|   |--- UBAND_DIFF_TECHNICAL_REPORT.tex
|   |--- sections/
|   |   |--- introduction.tex
|   |   |--- design_philosophy.tex
|   |   |--- mathematical_foundation.tex
|   |   |--- discrimination_hierarchy.tex
|   |   |--- sub_lsb_detection.tex
|   |   |--- implementation.tex
|   |   |--- validation.tex
|   |   |--- usage_examples.tex
|   |   |--- conclusion.tex
|   |   |--- appendix_thresholds.tex
|   |   `--- appendix_code_structure.tex
|   |--- DISCRIMINATION_HIERARCHY.md
|   |--- SUB_LSB_DETECTION.md
|   `--- IMPLEMENTATION_SUMMARY.md
`--- makefile
\end{verbatim}

\subsection{Key Function Reference}

\subsubsection{DifferenceAnalyzer Class}

\begin{lstlisting}[language=C++]
class DifferenceAnalyzer {
public:
    void process_difference(const ColumnValues& vals);

private:
    void process_raw_values(const ColumnValues& vals);
    void process_rounded_values(const ColumnValues& vals);

    double round_to_decimals(double value, int dp);
    bool is_sub_lsb_difference(double raw_diff, double big_zero);

    Thresholds thresh;
    CountStats counter;
    DiffStats differ;
    Flags flags;
};
\end{lstlisting}

\subsubsection{FileComparator Class}

\begin{lstlisting}[language=C++]
class FileComparator {
public:
    bool compare_files();

private:
    bool process_line(const string& line1, const string& line2);
    void process_column(size_t col, const string& str1, const string& str2);
    void process_difference(const ColumnValues& vals);

    void print_diff_like_summary();
    void print_rounded_summary();
    void print_significant_summary();

    FileReader reader1, reader2;
    DifferenceAnalyzer analyzer;
};
\end{lstlisting}

\subsubsection{FileReader Class}

\begin{lstlisting}[language=C++]
class FileReader {
public:
    bool open(const string& filename);
    bool read_line(string& line);

    bool is_first_column_range_data();
    bool is_first_column_monotonic();
    bool is_first_column_fixed_delta();

private:
    ifstream file;
    vector<double> first_column_values;
    bool first_column_analyzed;
};
\end{lstlisting}

\subsection{Build System}

\subsubsection{Makefile Targets}

\begin{description}
    \item[\texttt{make all}] Build main executable and tests
    \item[\texttt{make uband\_diff}] Build main executable only
    \item[\texttt{make tests}] Build test suite
    \item[\texttt{make test}] Run all unit tests
    \item[\texttt{make clean}] Remove build artifacts
    \item[\texttt{make help}] Display available targets
\end{description}

\subsubsection{Compiler Flags}

\begin{lstlisting}[language=make]
CXX = g++
CXXFLAGS = -std=c++17 -Wall -Wextra -pedantic -O2
LDFLAGS = -lm
\end{lstlisting}

\subsection{Dependency Graph}

\begin{verbatim}
uband_diff (main)
  `-- FileComparator
      |-- FileReader
      |   `-- LineParser
      |       `-- FormatTracker
      `-- DifferenceAnalyzer
          `-- (Thresholds, CountStats, DiffStats, Flags)

test_main
  |-- test_sub_lsb_boundary
  |-- test_percent_threshold
  |-- test_sensitive_threshold
  |-- test_semantic_invariants
  `-- (all tests link against same components)
\end{verbatim}

\subsection{Testing Infrastructure}

\subsubsection{Unit Test Framework}

Uses lightweight custom framework (no external dependencies):

\begin{lstlisting}[language=C++]
#define ASSERT_TRUE(condition) \
    if (!(condition)) { \
        std::cerr << "FAILED: " << #condition << std::endl; \
        return false; \
    }

#define ASSERT_EQ(actual, expected) \
    if ((actual) != (expected)) { \
        std::cerr << "FAILED: " << #actual << " != " << #expected \
                  << " (" << (actual) << " != " << (expected) << ")" \
                  << std::endl; \
        return false; \
    }
\end{lstlisting}

\subsubsection{Test Execution}

\begin{lstlisting}[language=bash]
$ make test
g++ -std=c++17 -o build/bin/test_main tests/*.cpp src/*.cpp
Running unit tests...
[PASS] test_sub_lsb_exact_half
[PASS] test_sub_lsb_multiple_precision
[PASS] test_percent_threshold_activation
...
[==========] 43 tests passed, 0 failed
\end{lstlisting}


\end{document}
